%!TEX root = ./template-skripsi.tex
%-------------------------------------------------------------------------------
%               		ABSTRAK
%-------------------------------------------------------------------------------
\begin{abstractind}

\textbf{HAFIZHUN ALIM}. \textit{Fish Movement Tracking} Mengguanakan Metode \textit{Gaussian Mixture Models} dan \textit{Kalman Filter}. Skripsi. Fakultas Matematika dan Ilmu Pengetahuan Alam Universitas Negeri Jakarta. 2023. Dibawah bimbingan Drs. Mulyono, M.Kom dan Muhammad Eka Suryana, M.Kom.\\

Potensi industri perikanan Indonesia merupakan yang terbesar di dunia, baik perikanan tangkap maupun perikanan budidaya. Salah satu masalah yang sering dihadapi oleh pelaku usaha budidaya ikan adalah pada saat proses penghitungan ikan yang masih menggunakan cara-cara manual. Penelitian ini bertujuan untuk melakukan proses penghitungan ikan dengan lebih mudah dan efisien. Dalam penelitian ini dilakukan proses \textit{tracking} objek ikan menggunakan \textit{Kalman Filter} yang dibantu oleh GMM sebagai metode deteksinya. Keluaran yang didapat menunjukan bahwa sistem mampu melakukan pelacakan dengan \textit{error} yang kecil, serta menghasilkan rata-rata jumlah objek yang mendekati rata-rata jumlah objek sebenarnya pada video kategori latar belakang sederhana.

\vspace{0.5cm}
\noindent
\textbf{Kata kunci :} \emph{Object Detection} \emph{Fish Movement Tracking}, Ikan, \emph{Object Tracking}, Peternakan Ikan. \\

\end{abstractind}

