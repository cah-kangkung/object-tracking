\begin{abstracteng}
	
	\textbf{HAFIZHUN ALIM}. \textit{Fish Movement Tracking} Using \textit{Gaussian Mixture Models} and \textit{Kalman Filter}. Thesis. Computer Science Study Program, Faculty of Mathematics and Natural Sciences, 
	State University of Jakarta. January 2023. Under the supervision of Drs. Mulyono, M.Kom dan Muhammad Eka Suryana, M.Kom.\\
	
	\textit{The potential of Indonesia's fisheries industry is the largest in the world, both capture fisheries and aquaculture. One of the problems often faced by fish farmers is during the fish counting process which still uses manual methods. This research aims to make the fish counting process easier and more efficient. In this research, the process of \textit{tracking} fish objects using \textit{Kalman Filter} assisted by GMM as a detection method is carried out. The output obtained shows that the system is able to track with a small error, as well as produce an average number of objects that is close to the average number of actual objects in the simple background video category.}
	
	\vspace{0.5cm}
	\noindent
	\textbf{\textit{Keywords :}} \emph{Object Detection} \emph{Fish Movement Tracking}, \emph{Fish}, \emph{Object Tracking}, \emph{Fishery}. \\
	
\end{abstracteng}